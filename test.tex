%#! luatex
\input luatexja.sty
\input lua-visual-debug.sty
\jfont \jSK="name:Source Han Serif SC:jfm=xjsh/kaiming" at 10pt
\jfont \jSQ="name:Source Han Serif SC:jfm=xjsh/quanjiao" at 10pt
\jfont \jSB="name:Source Han Serif SC:jfm=xjsh" at 10pt

\jfont \jTK="name:Source Han Serif TC:jfm=xjth/kaiming" at 10pt
\jfont \jTQ="name:Source Han Serif TC:jfm=xjth/quanjiao" at 10pt
\jfont \jTB="name:Source Han Serif TC:jfm=xjth" at 10pt

\parindent=10pt
\hsize=250pt

% text: modified from CLREQ
\def\testS#1{\par\noindent{\tentt\string#1}\par{#1\rightskip0pt plus 1fil
  一个简体字可能对应多?个繁体字,如简体字「发」,其相应的!〈繁体字可能为「發」或「髮」;
  一个繁体汉字对应多个简体・〜〉汉字的情。,况与前者相比数量极少但仍需注意,如繁体字「乾」可能对应简体字「干」或「乾」。
  繁简汉字的对应关系具体应由上下文决定。\par}\medskip}

\def\testT#1{\par\noindent{\tentt\string#1}\par{#1\rightskip0pt plus 1fil
  一個簡體字可能對應多?個繁體字,如簡體字「发」,其相應的!〈繁體字可能為「發」或「髮」;
  一個繁體漢字對應多個簡體・〜〉漢字的情。,況與前者相比數量極少但仍需注意,如繁體字「乾」可能對應簡體字「干」或「乾」。
  繁簡漢字的對應關係具體應由上下文決定。\par}\medskip}



\testS\jSK
\testS\jSQ
\testS\jSB

\testT\jTK
\testT\jTQ
\testT\jTB
\bye